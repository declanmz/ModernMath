\documentclass[12pt]{article}

\usepackage[margin=1.0in]{geometry}
\usepackage{sectsty}
% \usepackage{indentfirst}
\usepackage{amsmath}
\usepackage{dsfont}

\sectionfont{\fontsize{12}{12}\selectfont}
\subsectionfont{\fontsize{12}{12}\selectfont}
\renewcommand{\thesubsection}{\thesection.\alph{subsection}}

\begin{document}

\title{Group Theory 3}
\author{Declan Murphy Zink}
\date{2/25/2021}
\maketitle

\setcounter{section}{0}

\section{If aH and bH are distinct left cosets of H in G, are Ha and Hb distinct right cosets of H in G? Prove that this is true or give a counterexample.}
This is false. Counterexample:\\\\
In $D_6$, let $H = \{e,a\}$. Thus $bH = \{b,ab^2\}$ and $abH = \{ab, b^2\}$, which are distinct left cosets. However $Hb = \{b,ab\}$ and $Hab = \{ab, b\}$, thus $Hb = Hab$, so this statement is false.

\section{If in G $a^5 = e$ and $aba^{-1} = b^2$, find the order of b if $b \neq e$.}
$$aba^{-1} = b^2$$
$$ab = b^2a$$
And...
$$a^5 = e$$
$$a^4 = a^{-1}$$
Then...
$$aba^{-1} = b^2$$
$$ba^{-1} = a^{-1}b^2$$
$$ba^4 = a^4b^2$$
$$ba^4 = aaaabb$$
$$ba^4 = aaab^2ab$$
$$ba^4 = aab^4aab$$
$$...$$
$$ba^4 = b^{32}a^4$$
$$b = b^{32}$$
$$e = b^{31}$$
Since 31 is a prime number, the order of b must be 31.
\end{document}