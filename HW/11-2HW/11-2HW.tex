\documentclass[12pt]{article}

\usepackage[margin=1.0in]{geometry}
\usepackage{sectsty}
% \usepackage{indentfirst}
\usepackage{amsmath}
\usepackage{dsfont}

\sectionfont{\fontsize{12}{12}\selectfont}
\subsectionfont{\fontsize{12}{12}\selectfont}
\renewcommand{\thesubsection}{\thesection.\alph{subsection}}

\begin{document}

\title{Assignment 20: Congruence}
\author{Declan Murphy Zink}
\date{11/2/2020}
\maketitle

\setcounter{section}{1}

\section{Let $p$ be a prime number and $k$ a positive integer.}
\subsection{Show that if $x$ is an integer such that $x^2 \equiv x $ mod $ p$, then $x \equiv 0$ or $1$ mod $p$.}
For some $a \in \mathds{Z}$:
\begin{align*}
    x^2 &= x + pa\\
    x^2 - x &= pa\\
    x(x-1) &= pa
\end{align*}
Thus either $p|x$ or $p|(x-1)$.
$p$ can only divide a multiple of itself (including 0),
so $x \equiv 0$ mod $p$ or $(x-1) \equiv 0$ mod $p \Rightarrow x \equiv 1$ mod $p$.

\subsection{Show that if $x$ is an integer such that $x^2 \equiv x$ mod $p^k$, then $x \equiv 0$ or $1$ mod $p^k$.}
For some $a \in \mathds{Z}$:
\begin{align*}
    x^2 &= x + p^ka\\
    x^2 - x &= p^ka\\
    x(x-1) &= p^ka
\end{align*}
Since $x$ and $x-1$ are coprime, $p$ cannot divide both.
Thus, either $p^k|x$ or $p^k|(x-1)$. 
$p^k$ can only divide a multiple of itself (including 0),
so $x \equiv 0$ mod $p^k$ or $(x-1) \equiv 0$ mod $p^k \Rightarrow x \equiv 1$ mod $p^k$.


\section{
    For each of the following congruence equations, either find a solution $x \in \mathds{Z}$ or show that no solution exists:
}
\subsection{$99x \equiv 18$ mod $30$.}
$99x = 18 + 30y, y \in \mathds{Z}$\\
$hcf(99,30) =$ ?
\begin{align*}
    99 &= 3(30) + 9\\
    30 &= 3(9) + 3\\
    9 &= 3(3) + 0
\end{align*}
$hcf(99,30) = 3$ \\
$\Rightarrow \exists$ $s,t \in \mathds{Z}$ such that $99s + 30t = 3$
\begin{align*}
    3 &= 30 - 3(9)\\
    3 &= 30 - 3(99 - 3(30))\\
    3 &= -3(99) + 10(30)
\end{align*}
Thus $99(-3) = 3 + 30(-10)$\\
multiplying by 6: $99(-18) = 18 + 30(-60) \Rightarrow 99(-18) \equiv 18$ mod $30$.\\
$x = -18$

\subsection{$91x \equiv 84$ mod $143$.}
$hcf(91, 143) =$ ?
\begin{align*}
    143 &= 1(91) + 52\\
    91 &= 1(52) + 39\\
    51 &= 1(39) + 13\\
    39 &= 3(13) + 0
\end{align*}
$hcf(91, 143) = 13$\\
13 does not divide 84, so there is no solution.

\subsection{$x^2 \equiv 2$ mod $5$.}
$0^2 \equiv 0$ mod $5$\\
$1^2 \equiv 1$ mod $5$\\
$2^2 \equiv 4$ mod $5$\\
$3^2 \equiv 4$ mod $5$\\
$4^2 \equiv 1$ mod $5$\\
No solution exists.

\subsection{$x^2 + x + 1 \equiv 0$ mod $5$.}
$0^2+0+1 \equiv 1$ mod $5$\\
$1^2+1+1 \equiv 3$ mod $5$\\
$2^2+2+1 \equiv 2$ mod $5$\\
$3^2+3+1 \equiv 3$ mod $5$\\
$4^2+4+1 \equiv 1$ mod $5$\\
No solution exists.

\subsection{$x^2 + x + 1 \equiv 0$ mod $7$.}
$x^2 + x \equiv -1$ mod $7$\\
$x(x+1) \equiv 6$ mod $7$\\
$x = 2$ is a solution.

\setcounter{section}{4}

\section{}
\subsection{
    Use the fact that 7 divides 1001 to find your own "rule of 7."
    Use your rule to work out the remainder when 6005004003002001 is divided by 7.
}
Since $7|1001$:\\
$10^3 \equiv -1$ mod $7$\\
$10^6 \equiv 1$ mod $7$\\
$10^9 \equiv -1$ mod $7$\\
so $10^{3n} \equiv (-1)^n$ mod $7$\\\\
Thus for $a_1 ... a_k \in \mathds{Z}$ and $i \in \mathds{Z}$:\\
$a_1(10^0) + a_2(10^3) + ... + a_k(10^{3i}) \equiv a_1 - a_2 + a_3 - a_4 + ... \pm a_k$ mod $7$.\\\\
So, $6005004003002001 \equiv 1-2+3-4+5-6 \equiv -3 \equiv 4$ mod $7$.\\
Thus the remainder is 4.

\subsection{
    13 also divides 1001. Use this to get a rule of 13 and find the remainder when 6005004003002001 is divided by 13.
}
This is the same rule as 7, so $6005004003002001 \equiv 1-2+3-4+5-6 \equiv -3 \equiv 10$ mod $13$.\\
Thus the remainder is 10.

\subsection{
    Use the observation that $27 \times 37 = 999$ to work out a rule of 37, and find the remainder when 6005004003002001 is divided by 37.
}
Since $27 \times 37 = 999$, therefore $37 | 999$, so:\\
$10^3 \equiv 1$ mod $37$\\
$10^6 \equiv 1$ mod $37$\\
so $10^{3n} \equiv 1$ mod $37$\\\\
Thus for $a_1 ... a_k \in \mathds{Z}$ and $i \in \mathds{Z}$:\\
$a_1(10^0) + a_2(10^3) + ... + a_k(10^{3i}) \equiv a_1 + a_2 + ... + a_k$ mod $37$.\\\\
So, $6005004003002001 \equiv 1+2+3+4+5+6 \equiv 21$ mod $37$.\\
Thus the remainder is 21.

\section{
    Let $p$ be a prime number, and let $a$ be an integer that is not divisible by $p$.
    Prove that the congruence equation $ax \equiv 1$ mod $p$ has a solution $x \in \mathds{Z}$.
}
Since $p$ is prime and $p$ doesn't divide $a$, $hcf(a,p) = 1$. Thus $\exists$ $s,t \in \mathds{Z}$ such that $as + pt = 1$.
Therefore $as = 1 - pt$, so $x=s$ is a solution in $\mathds{Z}$ to $ax \equiv 1$ mod $p$.

\end{document}