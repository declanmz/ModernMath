\documentclass[12pt]{article}

\usepackage[margin=1.0in]{geometry}
\usepackage{sectsty}
% \usepackage{indentfirst}
\usepackage{amsmath}
\usepackage{dsfont}

\sectionfont{\fontsize{12}{12}\selectfont}
\subsectionfont{\fontsize{12}{12}\selectfont}
\renewcommand{\thesubsection}{\thesection.\alph{subsection}}

\begin{document}

\title{Combinatorics and Counting 2}
\author{Declan Murphy Zink}
\date{12/7/2020}
\maketitle

\setcounter{section}{5}

\section{}
\subsection{
    Find the number of arrangements of the set $\{1,2,...,n\}$ in which the numbers $1,2$ appear as neighbours.
}
When arranging the elements of the set, let us assume for convienence that the position for 1 is chosen first, then 2, ..., then n.
There will always be n possible positions for 1, but for 2 there are two scenarios:
if 1 was place on either end (where 2 would only have 1 possible placement) or if 1 was not placed on either end (where 2 would have 2 possible placements).
The following numbers simply have (n-2)! arrangements.
Thus the total possible arrangements is such:
$$\frac{2}{n}((n)(n-2)!) + \frac{n-2}{n}((n)(2)(n-2)!)$$
$$2(n-2)! + 2(n-2)(n-2)!$$
$$(2+2(n-2))(n-2)!$$
$$2(n-1)(n-2)!$$
$$2(n-1)!$$

\subsection{
    Let $n \leq 5$. Find the number of arrangements of the set $\{1,2,...,n\}$ in which the numbers $1,2,3$ appear as neighbors
    in order, and so do numbers $4,5$.
}
Let us assume the position of 1,2,3 is chosen first, thus there will be (n-2) placements.
For 4,5 there are two scenarios: if 1,2,3 was placed on either end (where there would be n-4 possible placements)
or if 1,2,3 was not placed on either end (where there would be n-5 possible placements).
The following numbers simply have (n-5)! arrangements.
Thus the total possible arrangements is such:
$$(n-2) \left( \left( \frac{n-4}{n-2} \right) (n-5) + \left( \frac{2}{n-2} \right) (n-4) \right) (n-5)!$$
$$\left((n-5)(n-4)+2(n-4)\right)(n-5)!$$
$$(n-3)(n-4)(n-5)!$$
$$(n-3)!$$

\section{
    Liebeck has $n$ steaks and is surrounded by $n$ hungry wolves. He throws each of the steaks to a random wolf. What is the chance that:
}
\subsection{Every wolf gets a steak?}
Simply determine the chance that a steak will be thrown to a wolf that has not already eaten for each steak, then multiply the probabilities.
The first steak has an $\frac{n}{n}$ chance, the second has $\frac{n-1}{n}$, the third has $\frac{n-2}{n}$, and so on.
Multiplied together we get the chance that every wolf gets a steak as $\frac{n!}{n^n}$.

\subsection{Exactly one wolf does not get a steak?}
Determine the total number of possible arrangements where exactly one wolf doesn't get a steak and divide by the total possible arrangements.
For one wolf to not get a steak, exactly one wolf must get two steaks. View the steaks as a set of numbers which represent the order in which the steaks were tossed:
$\{1,2,...,n\}$. Thus the number of ways a single wolf can get two steaks is $\binom{n}{2}$.
The following wolves can be arranged in $P(n-1,n-2)$ ways (where one wolf does not get a steak).
Thus the total number of possible arrangements where exactly one wolf doesn't get a steak is $\binom{n}{2} P(n-1,n-2) = \binom{n}{2} (n-1)!$.
The total possible arrangements is $n^n$, so the chance that exactly one wolf doesn't get a steak is $\frac{\binom{n}{2} (n-1)!}{n^n}$.

\subsection{Liebeck gets eaten, in the case where $n=7$?}
Assume Liebeck gets eaten when at least one wolf does not get a steak. Thus the chance for him to get eaten is $1-\frac{n!}{n^n}$.
For $n=7$, this chance is .99388 or 99.388\%. 

\setcounter{section}{12}

\section{}
\subsection{Find the coefficient of $x^{15}$ in $(1+x)^{18}$.}
$\binom{18}{15} = 816$

\subsection{Find the coefficient of $x^4$ in $(2x^3 - \frac{1}{x^2})^8$.}
$(a+b)^n = \sum_{k=0}^{n} \binom{n}{k} a^{n-k} b^k$, so, for $(2x^3 - \frac{1}{x^2})^8$,
$x^4$ will occur when $k=4$. Thus the coefficient is $\binom{8}{4}(2)^4(-1)^4 = 1120$.

\subsection{Find the constant term in the expansion of $(y+x^2-\frac{1}{xy})^{10}$.}
Find a k and j such that $(y)^k (x^2)^j (\frac{1}{xy})^{10-k-j} = 1$. For this constant term, $k=4, j=2$.
Thus this means the constant term is $\binom{10}{4,2,4}(1)^4(1)^2(-1)^4 = 3150$.

\setcounter{section}{15}
\section{
    At a party with six rather decisive people, any two people either like each other or dislike eachother.
    Prove that at this party, either (i) there are three people all of whom like each other, or (ii) there are three people,
    all of whom dislike eachother. Show that it is possible to have a party with five (decisive) people where neither (i) nor (ii) holds.
}
Imagine the party as a graph with six points A, B, C, D, E, F and lines connecting every point.
Let us represent a green line as liking eachother and a red line as disliking eachother.
Looking at the point A, there are two possible scenarios for the five lines connected to it: either at least 3 lines are green or at least 3 lines are red.
Let us look at if at least three lines are green and say they connect to B, C, and D.
If there is a green line between any one of B, C, or D, then we have scenario (i), where three people like eachother: 
A and the the two out of B, C, and D that are connected with a green line.
However, if there is not a green line between any one of B, C, or D, then we have scenario (ii), where three people dislike eachother:
B, C, and D.
This scenario looking at point A is the same if at least 3 lines connected to it are red (except that the red and green lines are swapped).
Thus, for 6 people, either (i) or (ii) must be true.\\\\
For five people this is not the case. Counterexample: Make a graph with five points A, B, C, D, E.
If green lines only connect A-B, B-C, C-D, D-E, and E-A, and red lines connect everything else, then there are no three
people who mutually like each other or mutually dislike eachother.


\end{document}