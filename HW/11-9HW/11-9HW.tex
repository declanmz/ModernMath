\documentclass[12pt]{article}

\usepackage[margin=1.0in]{geometry}
\usepackage{sectsty}
% \usepackage{indentfirst}
\usepackage{amsmath}
\usepackage{dsfont}

\sectionfont{\fontsize{12}{12}\selectfont}
\subsectionfont{\fontsize{12}{12}\selectfont}
\renewcommand{\thesubsection}{\thesection.\alph{subsection}}

\begin{document}

\title{Assignment 21: Congruence 4}
\author{Declan Murphy Zink}
\date{11/9/2020}
\maketitle

\setcounter{section}{5}

\section{
    The number $(p-1)!$ (mod p) came up in our proof of Fermat's Little Theorem, although we didn't need to find it.
    Calculate $(p-1)!$ (mod p) for some small prime numbers $p$. Find a pattern and make a conjecture.
    Prove your conjecture!
}
$p=2 \Rightarrow (p-1)! \equiv 1 \equiv -1$ mod 2\\
$p=3 \Rightarrow (p-1)! \equiv 2 \equiv -1$ mod 3\\
$p=5 \Rightarrow (p-1)! \equiv 24 \equiv -1$ mod 5\\
$p=7 \Rightarrow (p-1)! \equiv 720 \equiv -1$ mod 7\\\\
Conjecture: For a prime number $p$, $(p-1)! \equiv -1$ mod p\\\\
Proof:\\
$(p-1)! = (1)(2)(3)...(p-3)(p-2)(p-1)$\\
Let $a_1 = 1,..., a_{p-1} = p-1$\\
Since $p$ is prime we know that any $a_i$ must be coprime to $p$.\\
This means $\forall$ $a_i$ $\exists$ $x \in \mathds{Z}_p$ such that $a_i x \equiv 1$ mod p.
This $x$ is unique in $\mathds{Z}_p$.\\
For 1 and $p-1$, $x$ 

\end{document}