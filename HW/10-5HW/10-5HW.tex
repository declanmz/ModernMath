\documentclass[12pt]{article}

\usepackage[margin=1.0in]{geometry}
\usepackage{sectsty}
% \usepackage{indentfirst}
\usepackage{amsmath}
\usepackage{dsfont}

\sectionfont{\fontsize{12}{12}\selectfont}
\subsectionfont{\fontsize{12}{12}\selectfont}
\renewcommand{\thesubsection}{\thesection.\alph{subsection}}

\begin{document}

\title{Assignment 8: Complex Numbers 1}
\author{Declan Murphy Zink}
\date{10/5/2020}
\maketitle

\setcounter{section}{2}

\section{}
\subsection{
    Find the real and imaginary parts of $(\sqrt{3} - i)^{10}$ and $(\sqrt{3} - i)^{-7}$.
    For which values of $n$ is $(\sqrt{3} - i)^n$ real?
}
$(\sqrt{3} - i)^{10} = (2e^{- \frac{\pi}{6} i})^{10} = 2^{10} e^{- \frac{10 \pi}{6} i} = 2^9 \cdotp 2e^{\frac{\pi}{3} i} = 2^9(1 + \sqrt{3} i)$. \\ \\
$(\sqrt{3} - i)^{-7} = 2^{-7} e^{\frac{7 \pi}{6} i} = 2^{-8}(-\sqrt{3} -i)$. \\ \\
for $(\sqrt{3} - i)^{n}$ to be real, $\frac{n \pi}{6} = k \pi, k \in \mathds{Z}$, so n must be a multiple of 6.

\subsection{What is $\sqrt{i}$}
$\sqrt{i} = (e^{\frac{\pi}{2} i})^\frac{1}{2} = e^{\frac{\pi}{4} i}$ or $e^{\frac{5 \pi}{4} i}$

\subsection{
    Find all tenth roots of $i$. Which one is nearest to $i$ in the Argand diagram?
}
$\sqrt[10]{i} = (e^{\frac{\pi}{2} i})^\frac{1}{10} = e^{\frac{\pi}{20} i}, e^{\frac{5\pi}{20} i}, e^{\frac{9\pi}{20} i}, e^{13\frac{\pi}{20} i}, e^{17\frac{\pi}{20} i}, e^{21\frac{\pi}{20} i}, e^{25\frac{\pi}{20} i}, e^{29\frac{\pi}{20} i}, e^{33\frac{\pi}{20} i}, e^{37\frac{\pi}{20} i}.$ \\ \\
$e^{\frac{9\pi}{20} i}$ is closest to $i$.

\subsection{
    Find the seven roots of the equation $z^7 - \sqrt{3} + i = 0$. Which one of these roots is closest to the imaginary axis?
}
$z^7 = \sqrt{3} - i$,
so for $ k \in \mathds{Z}, 0 \leq k < 7, z = (2e^{(- \frac{\pi}{6} + 2k \pi) i})^{\frac{1}{7}} = 2^{\frac{1}{7}}e^{(- \frac{\pi}{42} + \frac{12k \pi}{42}) i}$ \\ \\
Closest to the imaginary axis = $2^{\frac{1}{7}}e^{\frac{23 \pi}{42} i}$.

\section{
    Prove the "Triangle Inequality" for complex numbers: $|u+v| \leq |u| + |v|$ for all $u,v \in \mathds{C}$.
}
For $a,b,c,d \in \mathds{R}$:
$$|a+bi + c+di| \leq |a+bi| + |c+di|$$
$$\sqrt{(a+c)^2 + (b+d)^2} \leq \sqrt{a^2 + b^2} + \sqrt{c^2 +  d^2}$$
$$(a+c)^2 + (b+d)^2 \leq a^2 + b^2 + c^2 + d^2 + 2\sqrt{(a^2 + b^2)(c^2 +  d^2)}$$
$$2ac + 2bd \leq 2\sqrt{(a^2 + b^2)(c^2 +  d^2)}$$
$$ac + bd \leq \sqrt{a^2c^2 + a^2d^2 + b^2c^2 + b^2d^2}$$
$$a^2c^2 + 2abcd + b^2d^2 \leq a^2c^2 + a^2d^2 + b^2c^2 + b^2d^2$$
$$0 \leq a^2d^2 - 2abcd + b^2c^2$$
$$0 \leq (ad - bc)^2$$
This must be true since $a,b,c,d \in \mathds{R}$.

\section{
    Let $z$ be a non-zero complex number. Prove that the three cube roots of $z$ are the corners of an equilateral triangle in the Argand diagram.
}
Let $z = re^{(\theta + 2k \pi) i}$ for $k \in \mathds{Z}, 0 \leq k < 3$.
Thus $\sqrt[3]{z} = r^{\frac{1}{3}}e^{(\frac{\theta}{3} + \frac{2k \pi}{3}) i}$,
which means that each cubic root is $\frac{2 \pi}{3}$ radians from the other two roots.
Since the modulus of each root is the same, $r^{\frac{1}{3}}$, this forms an equilateral triangle.



\end{document}