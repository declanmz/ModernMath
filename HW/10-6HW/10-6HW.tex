\documentclass[12pt]{article}

\usepackage[margin=1.0in]{geometry}
\usepackage{sectsty}
% \usepackage{indentfirst}
\usepackage{amsmath}
\usepackage{dsfont}

\sectionfont{\fontsize{12}{12}\selectfont}
\subsectionfont{\fontsize{12}{12}\selectfont}
\renewcommand{\thesubsection}{\thesection.\alph{subsection}}

\begin{document}

\title{Assignment 10: Complex Numbers 2}
\author{Declan Murphy Zink}
\date{10/6/2020}
\maketitle

\setcounter{section}{5}

\section{
    Express $\frac{1+i}{\sqrt{3} + i}$ in the form $x + iy$, where $x,y \in \mathds{R}$. 
    By writing each of $1+i$ and $\sqrt{3} + i$ in polar from, deduce that
    $$\cos\frac{\pi}{12} = \frac{\sqrt{3} + 1}{2 \sqrt{2}}, \; \sin\frac{\pi}{12} = \frac{\sqrt{3} - 1}{2 \sqrt{2}}.$$
}
Part 1:
$$\frac{1+i}{\sqrt{3} + i} = \frac{\sqrt{2} e^{\frac{\pi}{4} i}}{2 e^{\frac{\pi}{6} i}} = \frac{\sqrt{2}}{2} e^{\frac{\pi}{12} i} = \frac{\sqrt{2}}{2} (\cos\frac{\pi}{12} + i \sin\frac{\pi}{12})$$
Part 2:
$$\frac{1+i}{\sqrt{3} + i} \cdotp \frac{\sqrt{3} - i}{\sqrt{3} - i} = \frac{\sqrt{2}}{2} (\cos\frac{\pi}{12} + i \sin\frac{\pi}{12})$$
$$\frac{\sqrt{3} + \sqrt{3}i - i + 1}{3 + 1} \cdotp \frac{2}{\sqrt{2}} = \cos\frac{\pi}{12} + i \sin\frac{\pi}{12}$$
$$\frac{\sqrt{3} + 1}{2 \sqrt{2}} + i \frac{\sqrt{3} - 1}{2 \sqrt{2}} = \cos\frac{\pi}{12} + i \sin\frac{\pi}{12}$$
$$\cos\frac{\pi}{12} = \frac{\sqrt{3} + 1}{2 \sqrt{2}}, \; \sin\frac{\pi}{12} = \frac{\sqrt{3} - 1}{2 \sqrt{2}}$$

\section{}
\subsection{
    Show that $x^5 - 1 = (x-1)(x^4 + x^3 + x^2 + x + 1)$. 
    Deduce that if $\omega = e^{\frac{2 \pi i}{5}}$ then $\omega ^4 + \omega ^3 + \omega ^2 + \omega + 1 = 0$.
}
Part 1:
$$(x-1)(x^4 + x^3 + x^2 + x + 1) = x^5 + x^4 + x^3 + x^2 + x - x^4 - x^3 - x^2 - x - 1 = x^5 - 1$$
Part 2:\\
$\omega ^5 = 1$, so $\omega ^5 - 1 = 0 = (\omega - 1)(\omega ^4 + \omega ^3 + \omega ^2 + \omega + 1)$.
Thus, since $\omega \neq 1$, $\omega ^4 + \omega ^3 + \omega ^2 + \omega + 1 = 0$.

\subsection{
    Let $\alpha = 2\cos\frac{2 \pi}{5}$ and $\beta = 2\cos\frac{4 \pi}{5}$.
    Show that $\alpha = \omega + \omega ^4$ and $\beta = \omega ^2 + \omega ^3$.
    Find a quadratic equation with roots $\alpha, \beta$. Hence show that
    $$\cos\frac{2 \pi}{5} = \frac{1}{4}(\sqrt{5}-1)$$
}
$\omega ^4 = e^{\frac{8 \pi i}{5}} = e^{\frac{-2 \pi i}{5}} = \bar{\omega}$ 
and $\omega + \bar{\omega} = 2\cos\frac{2 \pi}{5}$, so $\omega + \omega ^4 = \alpha$.\\
$\omega ^3 = e^{\frac{6 \pi i}{5}} = e^{\frac{-4 \pi i}{5}} = \bar{(\omega ^2)}$ 
and $\omega ^2 + \bar{(\omega ^2)} = 2\cos\frac{4 \pi}{5}$, so $\omega ^2 + \omega ^3 = \beta$.\\\\

Quadratic equation: $(x - \alpha)(x - \beta) = x^2 - x(\alpha + \beta) + \alpha \beta = 0$.
\begin{itemize}
    \item $\alpha + \beta = \omega ^4 + \omega ^3 + \omega ^2 + \omega = -1$
    \item $\alpha \beta = (\omega + \omega ^4)(\omega ^2 + \omega ^3) = \omega ^3 + \omega ^4 + \omega ^6 + \omega ^7 = \omega ^4 + \omega ^3 + \omega ^2 + \omega = -1$
\end{itemize}
So, $\alpha, \beta$ are the roots of $x^2 + x - 1 = 0$, which, with the quadratic equation, has roots $\frac{1}{2}(-1 \pm \sqrt{5})$.
Since $\alpha = 2\cos\frac{2 \pi}{5} > 0$, $\cos\frac{2 \pi}{5} = \frac{1}{4}(-1+\sqrt{5})$.

\section{
    Find a formula for $\cos 4 \theta$ in terms of $\cos \theta$.
    Hence write down a quartic equation (i.e., an equation of degree 4) that has $\cos \frac{\pi}{12}$ as a root.
    What are the other roots of your equation?
}
Part 1:
$$\cos 4 \theta + i \sin 4 \theta = (\cos \theta + i \sin \theta)^4$$
Let $c = \cos \theta, s = \sin \theta$
$$\cos 4 \theta + i \sin 4 \theta = c^4 - 6c^2s^2 + s^4 + i(4c^3s - 4cs^3)$$
$$\cos 4 \theta = c^4 - 6c^2s^2 + s^4$$
$$\cos 4 \theta = c^4 - 6c^2(1-c^2) + (1-c^2)^2$$
$$\cos 4 \theta = 8\cos ^4 \theta - 8\cos ^2 \theta + 1$$
Part 2:
$$8(\cos\frac{\pi}{12})^4 - 8(\cos\frac{\pi}{12})^2 + 1 = \cos(4 \cdotp \frac{\pi}{12}) = \frac{1}{2}$$
$$16(\cos\frac{\pi}{12})^4 - 16(\cos\frac{\pi}{12})^2 + 2 = 1$$
So a quartic equation is $16x^4 - 16x^2 + 1 = 0$.\\\\
Part 3:\\
If $\cos(\alpha)$ is a root, then $\cos 4 \alpha = \frac{1}{2}$, so $\alpha = \frac{\pi}{12}, \; \frac{\pi}{12} + \frac{\pi}{2}, \; \frac{\pi}{12}+ \pi, \; \frac{\pi}{12} + \frac{3 \pi}{2}$.
So the roots of the quartic equation are $\cos\frac{\pi}{12}, \; \cos(\frac{\pi}{12} + \frac{\pi}{2}), \; \cos(\frac{\pi}{12}+ \pi), \; \cos(\frac{\pi}{12} + \frac{3 \pi}{2})$.
These can also be written as $\cos\frac{\pi}{12}, \; -\cos\frac{\pi}{12}, \; \sin\frac{\pi}{12}, \; -\sin\frac{\pi}{12}$.




\end{document}