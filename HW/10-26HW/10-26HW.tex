\documentclass[12pt]{article}

\usepackage[margin=1.0in]{geometry}
\usepackage{sectsty}
% \usepackage{indentfirst}
\usepackage{amsmath}
\usepackage{dsfont}

\sectionfont{\fontsize{12}{12}\selectfont}
\subsectionfont{\fontsize{12}{12}\selectfont}
\renewcommand{\thesubsection}{\thesection.\alph{subsection}}

\begin{document}

\title{Assignment 18: Prime Factorization}
\author{Declan Murphy Zink}
\date{10/26/2020}
\maketitle

\setcounter{section}{1}

\section{}
\subsection{Which positive integers have exactly three positive divisors?}
Squared prime numbers have exactly three positive divisors. If the prime is $a$, then these divisors are $1$, $a$, and $a^2$.

\subsection{Which positive integers have exactly four positive divisors?}
Cubed prime numbers have exactly four positive divisors. If the prime is $a$, then these divisors are $1$, $a$, $a^2$, and $a^3$. \\\\
Also products of two distinct prime numbers have exactly four positive divisors. If the primes are $a$ and $b$,
then these divisors are $1$, $a$, $b$, and $ab$.

\section{
    Suppose $n \geq 2$ is an integer with the property that whenever a prime $p$ divides $n$, $p^2$ also divides $n$
    (i.e., all primes in the prime factorization of $n$ appear at least to the power 2).
    Prove that $n$ can be written as the product of a square and a cube.
}
Let $n = p_1^{s_1}...p_k^{s_k}, s_i \geq 2$. We can write $s_i$ as $2u_i + 3v_i$
for integers $u_i, v_i \geq 0$ since $2u_i + 3v_i$ can produce all integers $\geq 2$.
(This can be proven because all even numbers $\geq 2$ can be created with $2u_i$ , 
all odd numbers $ \geq 2$ that are multiples of 3 can be created with $3v_i$,
and all odd numbers $ \geq 2$ that are not multiples of 3 can be created with $2u_i + 3(1)$.)
Thus $n = p_1^{2u_1 + 3v_1}...p_k^{2u_k + 3v_k} = (p_1^{2u_1}...p_k^{2u_k})(p_1^{3v_1}...p_k^{3v_k}) = (p_1^{u_1}...p_k^{u_k})^2(p_1^{v_1}...p_k^{v_k})^3$,
which is the product of a square and a cube.

\section{
    Prove that $lcm(a,b) = \frac{ab}{hcf(a,b)}$ for any positive integers $a,b$ without using prime factorization.
}
$\frac{ab}{hcf(a,b)}$ is a multiple of $a$ and $b$ because $\frac{ab}{hcf(a,b)} = a(\frac{b}{hcf(a,b)}) = b(\frac{a}{hcf(a,b)})$
and $\frac{b}{hcf(a,b)}$ and $\frac{a}{hcf(a,b)}$ are both integers.\\\\
Let $u$ be a multiple of $a$ and $b$ such that $u = \frac{ab}{s}, s \in \mathds{Z}$.
Thus, $\frac{u}{a} = \frac{b}{s} \Rightarrow s|b$ since $\frac{u}{a}$ is an integer.
Likewise $\frac{u}{b} = \frac{a}{s} \Rightarrow s|a$. Because $s|a$ and $s|b$, it must be true that $s|hcf(a,b)$.
Thus, the highest value possible for $s = hcf(a,b)$, so the lowest possible value for $u = \frac{ab}{hcf(a,b)}$.
Hence, $lcm(a,b) = \frac{ab}{hcf(a,b)}$.

\end{document}