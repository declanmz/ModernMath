\documentclass[12pt]{article}

\usepackage[margin=1.0in]{geometry}
\usepackage{sectsty}
% \usepackage{indentfirst}
\usepackage{amsmath}
\usepackage{dsfont}

\sectionfont{\fontsize{12}{12}\selectfont}
\subsectionfont{\fontsize{12}{12}\selectfont}
\renewcommand{\thesubsection}{\thesection.\alph{subsection}}

\begin{document}

\title{Functions 1}
\author{Declan Murphy Zink}
\date{12/10/2020}
\maketitle

\setcounter{section}{0}

\section{
    For each of the following functions $f$, say whether $f$ is 1-1 and whether $f$ is onto:
}
\subsection{$f$:$\mathds{R} \rightarrow \mathds{R}$ defined by $f(x)=x^2+2x$ for all $x \in \mathds{R}$.}
$f(x) = (x+1)^2 -1 \Rightarrow$ Neither 1-1 or onto. Image only spans $[-1, \infty)$, and $f(-2) = f(0)$.

\subsection{$f$:$\mathds{R} \rightarrow \mathds{R}$ defined by}
\[ f(x) = 
\begin{cases} 
    x-2 & x > 1\\
    -x & -1 \leq x \leq 1 \\
    x+2 & x < -1 
\end{cases}
\]
Onto, but not 1-1. Image spans $(-\infty, \infty) \Rightarrow \mathds{R}$, but $f(0) = f(2)$.

\subsection{$f$:$\mathds{Q} \rightarrow \mathds{R}$ defined by $f(x) = (x+ \sqrt{2})^2$.}

1-1 because in order for f(a) = f(b) for two rational nubmers a and b, either $a = b$ or $a+b = -2 \sqrt{2}:$\\
$$(a+\sqrt{2})^2 = (b+\sqrt{2})^2$$
$$a+\sqrt{2} = b+\sqrt{2} \; or \; a+\sqrt{2} = -b-\sqrt{2}$$
$$a = b \; or \; a+b = -2 \sqrt{2}$$
$a+b \neq -2 \sqrt{2}$ becuase they are rational, so it is 1-1.\\
Not onto because the image only spans $[0, \infty)$.

\subsection{$f: \mathds{N} \times \mathds{N} \times\mathds{N} \rightarrow \mathds{N}$ defined by $f(m,n,r) = 2^m 3^n 5^r$ for all $m,n,r \in \mathds{N}$.}
1-1 because 2,3, and 5 are prime factors. Not onto because Not all natural numbers can be created, such as 7.

\subsection{$f: \mathds{N} \times \mathds{N} \times\mathds{N} \rightarrow \mathds{N}$ defined by $f(m,n,r) = 2^m 3^n 6^r$ for all $m,n,r \in \mathds{N}$.}
Not 1-1 because $f(1,1,2) = f(2,2,1)$, ($2^m 3^n 6^r = 2^m 3^n 2^r 3^r$). Not onto because Not all natural numbers can be created, such as 7.

\subsection{
    Let $\sim$ be the equivalence relation on $\mathds{Z}$ defined by $a \sim b \Leftrightarrow a \equiv b$ mod $7$,
    and let $S$ be the set of equivalence classes of $\sim$. Define $f: S \rightarrow S$ by $f(cl(s)) = cl(s+1)$
    for all $s \in \mathds{Z}$.
}
$cl(0) \rightarrow cl(1) \rightarrow cl(2) \rightarrow cl(3) \rightarrow cl(4) \rightarrow cl(5) \rightarrow cl(6) \rightarrow cl(0)$,
so both onto and 1-1.

\setcounter{section}{2}

\section{
    Two functions $f, g : \mathds{R} \rightarrow \mathds{R}$ are such that for all $x \in \mathds{R}$, $g(x) = x^2 + x + 3$, and
    $(g \circ f)(x) = x^2 -3x + 5.$ Find the possibilities for $f$.
}
$f(x) = y \Rightarrow y^2 + y + 3 = x^2 - 3x +5 \Rightarrow (y)(y+1) = (x-1)(x-2)$. After graphing,
$y = 1-x$ or $x-2$, so $f(x) \in \{1-x, x-2\}$.

\setcounter{section}{5}

\section{}
\subsection{Find an onto function from $\mathds{N}$ to $\mathds{Z}$.}
\[ f(x) = 
\begin{cases} 
    -\frac{x-1}{2} & x \equiv 1 \; mod \; 2\\
    \frac{x}{2} & x \equiv 0 \; mod \; 2
\end{cases}
\]
Or: $f(x) = (-1)^x (\frac{x - (x \% 2)}{2})$, but I'm not sure if the modulus operator is allowed in pure math

\subsection{Find a 1-1 function from $\mathds{Z}$ to $\mathds{N}$.}
$f(x) = x^2 + x + |x|$

\section{}
\subsection{
    Let $S = \{1,2,3\}$ and $T = \{1,2,3,4,5\}$. How many functions are there from $S$ to $T$? How many of these are 1-1?
}
$5^3 = 125$ functions total.\\
$5*4*3 = 60$ 1-1 functions.

\subsection{
    Let $|S| = m, |T| = n$ with $m \leq n$. Show that the number of 1-1 functions from $S$ to $T$ is equal to $n(n-1)(n-2)...(n-m+1)$.
}
In a 1-1 function, each $s \in S$ must map to one $t \in T$. Thus, there are n choices for $s_1$, n-1 choices for $s_2$,
and so on, continuing until the last s ($s_m$), which will have n-m+1 choices. Thus the total 1-1 functions is $n(n-1)(n-2)...(n-m+1)$.

\end{document}