\documentclass[12pt]{article}

\usepackage[margin=1.0in]{geometry}
\usepackage{sectsty}
% \usepackage{indentfirst}
\usepackage{amsmath}
\usepackage{dsfont}

\sectionfont{\fontsize{12}{12}\selectfont}
\subsectionfont{\fontsize{12}{12}\selectfont}
\renewcommand{\thesubsection}{\thesection.\alph{subsection}}

\begin{document}

\title{Permuations 1}
\author{Declan Murphy Zink}
\date{2/4/2021}
\maketitle

\setcounter{section}{2}

\section{}
\subsection{List the numbers that occur as the orders of elements of $S_4$, and calculate how many elements there are in $S_4$ of each of these orders.}
Possible orders with corresponding cycle shapes and number of elements:\\
1: $(1^4)$ | 1\\
2: $(2^2), (2,1^2)$ | 9\\
3: $(3,1)$ | 8\\
4: $(4)$ | 6

\subsection{List all the possible cycle-shapes of even permuations in $S_6$.}
$(1^6), (2^2,1^2), (3,1^3), (3^2), (5,1), (4,2)$

\subsection{Calculate the largest possible order of any permutation in $S_{10}$.}
Cycle shape (2,3,5) = order 30.

\subsection{Calculate the largest possible order of any even permutation in $S_{10}$.}
Cycle shape (7,3) = order 21.

\subsection{Find the value of $n$ such that $S_n$ has an element of order greater than $n^2$.}
Cycle shape to get largest order is increasing primes, so cycle shape (2,3,5,7,11) does it, making $n=28$, since $2*3*5*7*11 = 2310 > 28^2 = 784$.

\setcounter{section}{4}

\section{Prove that exactly half of the $n!$ permutations in $S_n$ are even.}
Proved by creating a function $f$ that swaps even permutations to odd permutations and vice versa, then prove that $f$ is bijective, thus meaning that there are an equal amount of even and odd permutations for some $S_n$.\\\\
Proof:\\
$f(p) = p(1,2)$. Thus if $p$ is even, $f(p)$ is odd, and if $p$ is odd, $f(p)$ is even. $f(p)$ is injective because $f(p_1) = f(p_2) \Rightarrow p_1(1,2) = p_2(1,2) \Rightarrow p_1(1,2)(1,2) = p_2(1,2)(1,2) \Rightarrow p_1 = p_2$. $f(p)$ is surjective because $ \forall p \in S_n, f(p) = p(1,2)$ and $f(p(1,2)) = p(1,2)(1,2) = p$, thus for all permutations $y$ in the codomain of $f$, there exists a permutation $x$ in the domain of $f$ such that $f(x) = y$. Therefore, $f$ is bijective, meaning that for every even permutation there is exactly one odd permutation.


\end{document}