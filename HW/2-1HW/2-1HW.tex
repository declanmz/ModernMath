\documentclass[12pt]{article}

\usepackage[margin=1.0in]{geometry}
\usepackage{sectsty}
% \usepackage{indentfirst}
\usepackage{amsmath}
\usepackage{dsfont}

\sectionfont{\fontsize{12}{12}\selectfont}
\subsectionfont{\fontsize{12}{12}\selectfont}
\renewcommand{\thesubsection}{\thesection.\alph{subsection}}

\begin{document}

\title{More on Sets 1}
\author{Declan Murphy Zink}
\date{2/1/2021}
\maketitle

\setcounter{section}{1}

\section{Which of the following statements are true and which are false? Give proofs or counterexamples.}
\subsection{For any sets A, B, C, we have:}
$$A \cup (B \cap C) = (A \cup B) \cap (A \cup C)$$
True.\\
$x \in A \cup (B \cap C) \Leftrightarrow x \in A$ or $x \in B \cap C \Leftrightarrow x \in A \cup B$ and $x \in A \cup C \Leftrightarrow x \in (A \cup B) \cap (A \cup C)$

\subsection{For any sets A, B, C, we have:}
$$(A-B)-C = A-(B-C)$$
False.\\
If $A, B, C = \{1\}$, then $(A-B)-C = \emptyset$ and $A-(B-C) = \{1\}$.

\subsection{For any sets A, B, C, we have:}
$$(A-B) \cup (B-C) \cup (C-A) = A \cup B \cup C$$
False.\\
If $A, B, C = \{1\}$, then $(A-B) \cup (B-C) \cup (C-A) = \emptyset$ and $A \cup B \cup C = \{1\}$.

\setcounter{section}{10}
\section{Prove that if $m$ and $n$ are coprime positive integers, then $\phi(mn) = \phi(m)\phi(n)$}
$m = p_1^{a_1}...p_k^{a_k}$\\
$n = q_1^{b_1}...q_i^{b_i}$\\
Since $m$ and $n$ are coprime, $mn = p_1^{a_1}...p_k^{a_k}q_1^{b_1}...q_i^{b_i}$.\\\\
Thus, $\phi(m) = m(1-\frac{1}{p_1})...(1-\frac{1}{p_k})$\\
$\phi(n) = n(1-\frac{1}{q_1})...(1-\frac{1}{q_i})$\\
$\phi(mn) = mn(1-\frac{1}{p_1})...(1-\frac{1}{p_k})(1-\frac{1}{q_1})...(1-\frac{1}{q_i})$\\\\
Therefore, $\phi(mn) = \phi(m)\phi(n)$

\section{For a positive integer $n$, define\\ $$F(n) = \sum_{d|n} \phi(d)$$\\ where the sum is over the positive divisors $d$ of $n$, including both 1 and $n$. (For example, the positive divisors of 15 are 1, 3, 5, and 15.)}
\subsection{Calculate $F(15)$ and $F(100)$}
$F(15) = \phi(1) + \phi(3) + \phi(5) + \phi(15) = 15$\\
$F(100) = 100$

\subsection{Calculate $F(p^r)$, where $p$ is prime}
$F(p^r) = \phi(1) + \phi(p) + \phi(p^2) + ... + \phi(p^r)$\\
$F(p^r) = 1 + p(1-\frac{1}{p}) + p^2(1-\frac{1}{p}) + ... + p^r(1-\frac{1}{p})$\\
$F(p^r) = 1 + (p + ... + p^r)(1-\frac{1}{p})$\\
$F(p^r) = 1 + (p + ... + p^r) - (1 + p + ... p^{r-1})$\\
$F(p^r) = p^r$

\subsection{Calculate $F(pq)$, where $p,q$ are distinct primes.}
$F(pq) = \phi(1) + \phi(p) + \phi(q) + \phi(pq)$\\
$F(pq) = \phi(1) + \phi(p) + \phi(q) + \phi(p)\phi(q)$\\
$F(pq) = 1 + (p-1) + (q-1) + (p-1)(q-1)$\\
$F(pq) = 1 + p-1 + q-1 + pq - p - q + 1$\\
$F(pq) = pq$

\subsection{Formulate a conjecture about $F(n)$ for an arbitrary positive integer $n$. Try to prove your conjecture.}
Conjecture: $F(n) = n$.\\\\
For divisors $d$ such that $d|n$, $\phi(\frac{n}{d})$ gives the number of integers $x$ such that $gcd(x,n) = d$.
Thus, since each number less than or equal to $n$ has a gcd equal to a $d$, each integer less than or equal to $n$ is counted once, accumulating to $n$.




\end{document}